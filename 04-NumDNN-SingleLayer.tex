\documentclass[12pt,fleqn,handout]{beamer}


\xdefinecolor{lavendar}{rgb}{0.8,0.6,1}
\xdefinecolor{olive}{cmyk}{0.64,0,0.95,0.4}
%\xdefinecolor{olive}{cmyk}{1,0,0,0}
\xdefinecolor{mag}{cmyk}{0.1,1,0,0.2}
\xdefinecolor{lblue}{rgb}{0,0,1.5}
\xdefinecolor{lred}{rgb}{1,0,0}
\xdefinecolor{mine}{cmyk}{1,0,0.2,0}
\xdefinecolor{bluel}{cmyk}{0.1,0,0.9,0.4}

\usepackage{amsmath,amssymb,dsfont,mathrsfs}
\usepackage{tikz,pgflibraryplotmarks}
\usepackage{multimedia}
\usepackage{wasysym}
\usepackage{rotating}
\usepackage{algorithm,algorithmic}
\usepackage{graphicx} % more modern
\usepackage{subfigure}
\usepackage{booktabs}

\usepackage{pgfplots}
\usepackage{verbatim}

\usepackage{setspace}
\newlength\iwidth
\newlength\iheight

\newcommand\makebeamertitle{\frame{\maketitle}}%
\graphicspath{{./images/}}
\setbeamertemplate{navigation symbols}{}
\addtobeamertemplate{navigation symbols}{}{%
    \usebeamerfont{footline}%
    \usebeamercolor[fg]{footline}%
	\insertshorttitle
    \;--
    \insertframenumber
}

\newcommand{\sectionstart}{
	\only<beamer>{
 	\begin{frame}% (fold)
 		\begin{centering}\Huge \insertsection \par\end{centering}
 	\end{frame}% frame the_application (end)
	}
 }


% make bibliography entries smaller
\usepackage{natbib}
\setbeamertemplate{bibliography item}{[\theenumiv]}
\renewcommand\bibfont{\scriptsize}
\setbeamertemplate{frametitle continuation}[from second]
\newcommand{\tcr}{\textcolor{red}}
\newcommand{\tcrd}{\textcolor{red}}
\newcommand{\tcb}{\textcolor{bluel}}
\newcommand{\tcm}{\textcolor{mag}}
\newcommand{\tcg}{\textcolor{olive}}

\newcommand{\R}{\mathbb{R}}
\newcommand{\C}{\mathbb{C}}

% bold lower-case for vectors
\newcommand{\bfa}{{\bf a}}
\newcommand{\bfb}{{\bf b}}
\newcommand{\bfc}{{\bf c}}
\newcommand{\bfs}{{\bf s}}
\newcommand{\bfm}{{\bf m}}
\newcommand{\bfd}{{\bf d}}
\newcommand{\bfe}{{\bf e}}
\newcommand{\bfu}{{\bf u}}
\newcommand{\bfy}{{\bf y}}
\newcommand{\bfx}{{\bf x}}
\newcommand{\bfh}{{\bf h}}
\newcommand{\bfw}{{\bf w}}
\newcommand{\bfv}{{\bf v}}
\newcommand{\bfr}{{\bf r}}
\newcommand{\bfz}{{\bf z}}
\newcommand{\bfp}{{\bf p}}


% bold upper-case for linear operators
\newcommand{\bfA}{{\bf A}}
\newcommand{\bfB}{{\bf B}}
\newcommand{\bfZ}{{\bf Z}}
\newcommand{\bfM}{{\bf M}}
\newcommand{\bfC}{{\bf C}}
\newcommand{\bfD}{{\bf D}}
\newcommand{\bfQ}{{\bf Q}}
\newcommand{\bfJ}{{\bf J}}
\newcommand{\bfG}{{\bf G}}
\newcommand{\bfI}{{\bf I}}
\newcommand{\bfP}{{\bf P}}
\newcommand{\bfK}{{\bf K}}
\newcommand{\bfY}{{\bf Y}}
\newcommand{\bfW}{{\bf W}}
\newcommand{\bfR}{{\bf R}}
\newcommand{\bfL}{{\bf L}}
\newcommand{\bfF}{{\bf F}}
\newcommand{\bfT}{{\bf T}}
\newcommand{\bfS}{{\bf S}}
\newcommand{\bfX}{{\bf X}}
\newcommand{\bfU}{{\bf U}}
\newcommand{\bfV}{{\bf V}}
\newcommand{\bfH}{{\bf H}}

\newcommand{\diag}{\rm diag}

\newcommand{\calF}{\mathcal{F}}



\newcommand{\hf}{{\frac 12}}
\newcommand{\bftheta}{{\boldsymbol \theta}}
\newcommand{\bfmu}{{\boldsymbol \mu}}
\newcommand{\bfxi}{{\boldsymbol \xi}}

\newcommand{\bfLambda}{{\boldsymbol \Lambda}}
\newcommand{\bflambda}{{\boldsymbol \lambda}}
\newcommand{\bfSigma}{{\boldsymbol \Sigma}}
\newcommand{\bfepsilon}{{\boldsymbol \epsilon}}

\newcommand{\E}{\vec E}
\newcommand{\B}{\vec B}

\newcommand{\vu}{  {\vec {\bf u}}}

\newcommand{\grad}{  {\vec {\bf \nabla}}}

\newcommand{\lfrownie}{\textcolor{red}{\large{\frownie}}}
\newcommand{\lsmiley}{\textcolor{green}{\large{\smiley}}}

\newcommand{\curl}{\ensuremath{\nabla\times\,}}
\renewcommand{\div}{\nabla\cdot\,}
\newcommand{\divh}{\nabla_h\cdot\,}
\renewcommand{\grad}{\ensuremath{\nabla}}

\DeclareMathOperator*{\argmin}{arg\,min}


\title{ Single-Layer Neural Networks}
\subtitle{Numerical Methods for Deep Learning}
\date{
}
\begin{document}

\makebeamertitle

\section{Introduction}

\begin{frame}\frametitle{Motivation: Nonlinear Models}


In general, impossible to find a linear separator between classes
%$$ \bfC = \bfX \bfW + b $$

\begin{center}
	\begin{tabular}{cc}
		\includegraphics[width=45mm]{Circle-train} & 
		\invisible<beamer|1>{\includegraphics[width=45mm]{Circle-proptrain} }\\
		input features & \invisible<beamer|1>{transformed features}
	\end{tabular}
\end{center}

\bigskip

\invisible<beamer|1>{{\bf Goal/Trick}

Embed the points in higher dimension and/or move the points to make them
linearly separable}

\only<beamer|2>{}
\end{frame}

\begin{frame}
	\frametitle{Learning Objective: Single-Layer Neural Networks}
	
	In this module, we derive our first nonlinear model, i.e., a neural network with a single layer.
	
	\bigskip
	
	Learning tasks:
	\begin{itemize}
		\item classification $\leadsto$ multinomial logistic regression
		\item regression $ \leadsto$ nonlinear least-squares
	\end{itemize}
	
	\bigskip
	
	Numerical methods:
	\begin{itemize}
		\item Sample Average Approximation: Newton-CG, VarPro, \ldots
		\item Stochastic Optimization: SGD, ADAM, \ldots
	\end{itemize}
\end{frame}



\begin{frame}\frametitle{Example: Linear Regression}


Assume $\bfC\in \R^{n_c\times n}$, $\bfY \in \R^{n_f \times n}$ and $n \gg n_f$.
Goal: Find $\bfW \in \R^{n_c \times n_f}$ such that

$$ \bfC = \bfW \bfY $$

\bigskip
\pause

Since ${\rm rank}(\bfY)<n$, there will generally be no solution.

\bigskip
\pause

Two options:
\begin{enumerate}
	\item Regression: Solve $\min_\bfW \| \bfW \bfY - \bfC \|_F^2$ $\leadsto$ always has solutions, but residual might be large
	\item Nonlinear Model: Replace $\bfY$ by $\sigma(\bfK\bfY)$ in regression, where $\sigma$ is element-wise function (aka activation) and $\bfK \in \R^{m \times n_f}$ where $m \gg n_f$
\end{enumerate}

\end{frame}

\begin{frame}\frametitle{Illustrating Nonlinear Models}

\begin{center}
	\begin{tabular}{cc}
		\rotatebox{90}{original} & \includegraphics[width=.9\textwidth]{elmSmall}\\
		 \invisible<beamer|1>{\rotatebox{90}{transformed}} & 
		\invisible<beamer|1>{\includegraphics[width=.9\textwidth]{elmBig}}\\
	\end{tabular}
\end{center}

\bigskip

\invisible<beamer|1>{
Remarks
\begin{itemize}
	\item instead of $\bfW \bfY = \bfC$ solve $\hat{\bfW} \sigma(\bfK \bfY)  = \bfC$
	\item solve bigger problem $\leadsto$ memory, computation, \ldots
	\item what happens to ${\rm rank}(\sigma(\bfK\bfY))$ when $\sigma(x)=x$?
\end{itemize}}

\only<beamer|2>{}
\end{frame}

\begin{frame}\frametitle{Conjecture: Universal Approximation Properties}

Given the data $\bfY \in \R^{n_f \times n}$ and $\bfC \in \R^{n_c \times n}$
with $n\gg n_f$, there is a nonlinear function $\sigma:\R \to \R$, a matrix $\bfK \in \R^{m \times n_f}$, and a bias $\bfb \in \R^m$ such that

$$
 {\rm rank}(\sigma(\bfK \bfY + \bfb \bfe_n^\top)) = n.
$$

\bigskip
\pause
Therefore, possible to find ${\bfW}\in\R^{n_c\times m}$

$$\bfW \sigma( \bfK \bfY + \bfb\bfe_n^\top) = \bfC.$$

This is only a conjecture. For solid approximation theory see~\cite{Cybenko1989,HornikEtAl1989}.
\end{frame}


\begin{frame}\frametitle{Choosing Nonlinear Model}

$$ \bfW  \sigma(\bfK \bfY+ \bfb\bfe_n^\top)= \bfC $$
\begin{itemize}
\item how to choose $\sigma$?
\pause
\begin{itemize}
	\item early days: motivated by neurons
	\item popular choice: $\sigma(x) = \tanh(x)$ (smooth, bounded, \ldots)
	\item nowadays: $\sigma(x) = \max(x,0)$ (aka ReLU, rectified linear unit, non-differentiable, not bounded, simple)
\end{itemize}
\pause
\item how to choose $\bfK$ and $\bfb$?
\pause
\begin{itemize}
	\item pick randomly $\leadsto$ branded as \emph{extreme learning machines}~\cite{HuangEtAl2006}
	\item train (optimize) $\leadsto$ done for most neural network
	\item \emph{deep learning} when neural network has many layers
\end{itemize}
\end{itemize}
\end{frame}

\begin{frame}\frametitle{Extreme Learning Machines~\cite{HuangEtAl2006}}

Select activation function, choose $\bfK$ and $\bfb$ randomly, and solve the linear least-squares/classification problem.

\bigskip

Advantages:
\begin{itemize}
\item universal approximation theorem: can interpolate any function
\item very(!) easy to program, convex optimization
\item can serve as a benchmark to more sophisticated methods
\end{itemize}

\bigskip

Some concerns:
\begin{itemize}
\item may require very large $\bfK$ (scale with $n$, number of examples)
\item may not generalize well
\item large-scale optimization problem with no obvious structure
\end{itemize}

% \begin{center}
% \end{frame}
% 	\texttt{EELM\_Peaks.m}
% \end{center}
\end{frame}

\section{Objective Function}

\begin{frame}\frametitle{Today: Learning the Weights}


Why? Using random weights, $\bfK$ might need to be very large to fit training data (scales with $n$).

Also, solution may not generalize well to test data.

\bigskip
\pause

Idea: Learn $\bfK$ and $\bfb$  from the data (in addition to $\bfW$)

$$ \min_{\bfK,\bfW,\bfb} E(\bfW\sigma(\bfK \bfY + \bfb\bfe_n^\top), \bfC_{\rm obs}) + \lambda R(\bfW,\bfK,\bfb)$$

About this optimization problem:
\begin{itemize}
	\item unknowns $\bfW \in \R^{n_c \times m}$, $\bfK \in \R^{m \times n_f}$,  $\bfb \in \R^m$
	\item new hyper-parameter $m$ (aka width, number of neurons)
	\item  non-convex problem $\leadsto$ local minima, careful initialization
	\item need to compute derivatives w.r.t. $\bfK, \bfb$
\end{itemize}
\end{frame}


\begin{frame}
	\frametitle{Non-Convexity}
	The optimization problem is non-convex. Simple illustration of cross-entropy along two random directions $d\bfK$ and $d\bfW$

	\begin{center}
		\includegraphics[width=.6\textwidth]{nonConvexitySingleLayer}
		
		% (see \texttt{ESingleLayer\_PlotObjective.m})
	\bigskip
	
	Expect worse when number of layers grows!
	\end{center}

\end{frame}
% \begin{frame}[fragile]\frametitle{Training the Neural Network}
%
% \begin{itemize}
% \item If non-convexity is not ``too bad'' can use standard gradient based methods
% \item If non-convexity is ``ugly'' need to modify standard methods (stochastic kick)
% \item If non-convexity is ``bad'' need global optimization techniques
% \end{itemize}
%
% 	\begin{center}
% 	\begin{tabular}{ccc}
% 		\includegraphics[width=.3\textwidth]{images/goodConv} &
% 		\includegraphics[width=.3\textwidth]{images/badConv} &
% 		\includegraphics[width=.3\textwidth]{images/uglyConv} \\
% 		good & bad & ugly \\
% 		\end{tabular}
% 		\end{center}
%
%
%
% \end{frame}

\begin{frame}\frametitle{Recap: Differentiating Linear Algebra Expressions}

Easy ones:
\begin{align*}
 F_1(\bfx,\bfy) &= \bfx^{\top} \bfy  & \bfJ_{\bfx}F_1(\bfx,\bfy) = \bfy^\top\\
F_2(\bfA,\bfx)  &= \bfA \bfx         & \bfJ_{\bfx}F_2(\bfx,\bfy) = \bfA 
\end{align*}
\pause
For $\bfx = {\rm vec}(\bfX)$ what is
$$ F_3(\bfA,\bfX) =   \bfA  \bfX \quad \quad \bfJ_{\bfx} F_3 = ??? $$
\pause
Recall that
$${\rm vec}(\bfA \bfX) = {\rm vec}(\bfA  \bfX \bfI) = (\bfI \otimes \bfA) {\rm vec} (\bfX) $$
Therefore:
$$ \bfJ_{\bfx}F_3(\bfA,\bfX) = \bfI \otimes \bfA $$
\pause
\textcolor{red}{
Efficient mat-vec: } $\bfJ_{\bfX}F_3(\bfA,\bfX) \bfv = { \bfA\ \rm mat}(\bfv)$
	
\end{frame}


\begin{frame}\frametitle{Training Single Layer Neural Network}

Assume no regularization (easy to add) and re-write optimization problem as 
\begin{eqnarray*}
 \min_{\bfW,\bfK,\bfb}  E(\bfC_{\rm obs}, \bfZ,\bfW ) \quad \text{ with} \quad \bfZ = \sigma(\bfK\bfY + \bfb \bfe_n^\top)
\end{eqnarray*}

\bigskip
\pause

Agenda:
\begin{enumerate}
	\item compute derivative of $\bfz={\rm vec}(\bfZ)$ w.r.t. ${\rm vec}(\bfK), \bfb$ 
	\item use chain rule to get
	\begin{align*}
		\bfJ_{{\rm vec}(\bfK)} E  & = \bfJ_{{\rm vec}(\bfZ)} E(\bfC_{\rm obs},\bfZ,\bfW) \ \bfJ_{{\rm vec}(\bfK)} \bfZ\\
		\bfJ_{\bfb} E  & = \bfJ_{{\rm vec}(\bfZ)} E(\bfC_{\rm obs},\bfZ,\bfW) \ \bfJ_{\bfb} \bfZ
	\end{align*}
	\item efficient code for mat-vecs with $\bfJ$ and $\bfJ^\top$
\end{enumerate} 
\end{frame}

\begin{frame}\frametitle{Derivatives of a Single Layer Network}
$$
\bfZ = \sigma(\bfK\bfY + \bfb \bfe_n^\top)
$$
Recall that $\sigma$ is applied element-wise. Therefore
$$
	\bfJ_{{\rm vec}(\bfK)}\bfZ = {\rm diag}(\sigma'(\bfK \bfY + \bfb \bfe_n^\top)) (\bfY^\top \otimes \bfI)
$$
\pause
Efficient way to get matrix vector products
\begin{eqnarray*}
\bfJ_{\bfK}\bfZ \bfv &=& {\rm mat} \left({\rm diag}(\sigma'(\bfK \bfY + \bfb\bfe_n^\top)\  (\bfY^\top \otimes \bfI)\bfv\right)\\
                            &=& \sigma'(\bfK \bfY + \bfb \bfe_n^\top) \odot ( {\rm mat}(\bfv) \bfY )
\end{eqnarray*}
And for transpose
\begin{eqnarray*}
(\bfJ_{\bfK}\bfZ)^\top \bfu &=&  {\rm mat} \left( (\bfY \otimes \bfI) \ {\rm diag}(\sigma'(\bfK \bfY + \bfb\bfe_n^\top)) \bfu \right)\\
                            &=& (\sigma'(\bfK \bfY + \bfb\bfe_n^\top) \odot  {\rm mat}(\bfu))\  \bfY^\top
\end{eqnarray*}

\end{frame}




\begin{frame}[fragile]\frametitle{Coding Problem: Derivatives of Single Layer}

\textbf{Derivations:}
\begin{enumerate}
	
	\item compute $\bfJ_{\bfb}\bfZ \bfv$ and $(\bfJ_{\bfb}\bfZ)^\top \bfu$

	\item (optional) compute $\bfJ_{{\rm vec}(\bfY)}\bfZ  \bfv$ and $(\bfJ_{{\rm vec}(\bfY)}\bfZ)^\top  \bfu$
\end{enumerate}

\textbf{Coding:}
\begin{verbatim}
function[Z,JKt,Jbt,JYt,JK,Jb,JY] = singleLayer(K,b,Y)
% Returns Z = sigma(K*Y+b) and 
%                         functions for J'*U and J*V
\end{verbatim}

\textbf{Testing:}
\begin{enumerate}
	\item Derivative check for Jacobian mat-vec
	\item Adjoint tests for transpose, let $\bfv,\bfu$ be arbitray vectors
	$$
		\bfu^\top \bfJ \bfv \approx \bfv^\top \bfJ^\top \bfu
	$$
\end{enumerate}

\end{frame}


\begin{frame}[fragile]
	\frametitle{Putting Things Together}
	
	Implement loss function of single-layer NN
	$$
		E(\bfK,\bfb,\bfW) \stackrel{def}{=} E(\bfC,\bfZ,\bfW), \quad \bfZ = \sigma(\bfK \bfY+\bfb\bfe_n^\top)
	$$
	
	
	\begin{verbatim}
		function [Ec,dE] = singleLayerNNObjFun(x,Y,C,m)
		% where x = [K(:); b; W(:)]
		% evaluates single layer and computes cross entropy 
		%        and gradient (extend for approx. Hessian)
	\end{verbatim}

	\bigskip
	
    Use
	\begin{enumerate}
		\item $\nabla_{\bfZ} E =  \bfW^\top \nabla_{\bfS} \ E(\bfS), \quad \bfS = \bfW\bfZ$
		\item $\nabla_{\bfK} E =  \bfJ_{\bfK}^\top \nabla_{\bfZ} E$
		\item $\nabla_{\bfb} E =  \bfJ_{\bfb}^\top \nabla_{\bfZ} E$
		\item $\nabla_{\bfW} E =  \nabla_{\bfS} \ E(\bfS) \bfY $
	\end{enumerate}
	
	
\end{frame}




\section{Sample Average Approximation}

\begin{frame}
	\frametitle{Sample Average Approximation (SAA)}
	
	Note that the objective function in our learning problem is actually stochastic
	$$
	\frac{1}{n}E(\bfW \sigma(\bfK\bfY+\bfb\bfe_n^\top),\bfC_{\rm obs}) =  {\mathbb{E}}_{(\bfy,\bfc)}\left[E\left(\bfW \sigma(\bfK\bfy+\bfb),\bfc\right)\right]
	$$
	In general, $n$ will be too large to compute left hand side $\leadsto$ consider stochastic problem.
	
	\bigskip
	SAA idea:  Approximate expected value with relatively large sample $S \subset \{ 1,\ldots,n\}$. Use deterministic optimization method  
			$$
				\min_{\bfK,\bfb,\bfW}  \frac{1}{|S|} \sum_{s\in S} E\left(\bfW \sigma(\bfK\bfy+\bfb),\bfc\right).
			$$
		
			Pro: use your favorite solver, linesearch, stopping\ldots
			
			Con: large batches needed
			
			\begin{center}
				Note: Sample stays fixed during iteration, but occasional resampling recommended.
			\end{center}
\end{frame}

\begin{frame}
	\frametitle{Simple Option: BFGS, NLCG, \ldots}
	
	Since we have computed the gradient of our objective function, we can experiment with a wide range of methods already. 
	
	\bigskip
	
	Some candidates from \texttt{scipy.optimize.minimize} are:
	\begin{itemize}
		\item \texttt{CG} - nonlinear conjugate gradient
		\item \texttt{BFGS} 
		\item \texttt{Newton-CG} - attention: Hessian not spsd
		\item \texttt{trust-ncg} 
	\end{itemize}
	Note that for the latter two, Hessian mat-vecs will be approximated numerically (not very stable).
	
\end{frame}

\begin{frame}
	\frametitle{Better Option: Gauss-Newton Method}
	
	\textbf{Goal:} Use curvature information for fast convergence
	$$
	\nabla_{\bfK} E(\bfK,\bfb,\bfW) =  (\bfJ_\bfK \bfZ)^\top \nabla_\bfZ E(\bfW\sigma(\bfK \bfY+\bfb\bfe_n^\top),\bfC),
	$$
	where $\bfJ_\bfK \bfZ = \nabla_{\bfK} \sigma(\bfK \bfY+\bfb\bfe_n^\top)^\top$.\pause This means that Hessian is
	\begin{equation*}
		\begin{split}
	\nabla_{\bfK}^2 E(\bfK) &  = (\bfJ_{\bfK}\bfZ)^\top  \nabla_\bfZ^2 E(\bfC,\bfZ,\bfW) \bfJ_{\bfK}\bfZ\\
	 & + \sum_{i=1}^{n}\sum_{j=1}^{m} \nabla_{\bfK}^2 \sigma(\bfK \bfY+\bfb\bfe_n^\top)_{ij} \nabla_\bfZ E(\bfC,\bfZ,\bfW)_{ij}
		\end{split}
	\end{equation*}
	
	First term is spsd and we can compute it.
	
	\pause
	
	We neglect second term since
	\begin{itemize}
		\item can be indefinite and difficult to compute
		\item small if transformation is roughly linear or close to solution (easy to see for least-squares)
	\end{itemize}
	\begin{center}
		\textcolor{red}{same for $\bfb$ and use full Hessian for $\bfW$ $\leadsto$ ignore coupling!}
	\end{center}	
\end{frame}


\begin{frame}\frametitle{Even Better Option: Variable Projection~\cite{NewmanEtAl2020}}
	Idea: Treat learning problem as coupled optimization problem with blocks $\bftheta$ and $\bfW$. 
	
	Simple illustration for coupled least-squares problem~\cite{GoPe1973,GoPe03,OLearyRust2013}
	$$
		\min_{\bftheta,\bfw} J(\bftheta,\bfw) = \hf \| \bfA(\bftheta) \bfw - \bfc\|^2 + \frac{\lambda}{2}\| \bfL \bfw\|^2 + \frac{\beta}{2} \| \bfM \bftheta\|^2
	$$
	
	\pause
	
	Note that for given $\bftheta$ the problem becomes a standard least-squares problem. Define: 
	$$
		\bfw(\bftheta) = \left( \bfA(\bftheta)^\top \bfA(\bftheta) + \lambda \bfL^\top \bfL \right)^{-1}{\bfA(\bftheta)^\top \bfc}
	$$
	
	\pause
	This gives optimization problem in $\bftheta$ only (aka \emph{reduced/projected  problem})
	$$
		\min_{\bftheta} \tilde{J}(\bftheta) = \hf \| \bfA(\bftheta) \bfw(\bftheta) - \bfc\|^2 + \frac{\lambda}{2}\| \bfL \bfw(\bftheta)\|^2 + \frac{\beta}{2} \| \bfM \bftheta\|^2
	$$
	
\end{frame}
\begin{frame}\frametitle{ Variable Projection (cont.)}
	$$
		\min_{\bftheta} \tilde{J}(\bftheta) = \hf \| \bfA(\bftheta) \bfw(\bftheta) - \bfc\|^2 + \frac{\lambda}{2}\| \bfL \bfw(\bftheta)\|^2 + \frac{\beta}{2} \| \bfM \bftheta\|^2
	$$
	Necessary optimality condition:
	$$ 
		\nabla \tilde{J}(\bftheta) = \nabla_\bftheta J(\bftheta,\bfw) + \nabla_\bftheta \bfw(\bftheta) \nabla_{\bfw} J(\bftheta,\bfw) \stackrel{!}{=} 0.
	$$
	Less complicated than it seems since
	$$
		\nabla_{\bfw} J(\bftheta,\bfw(\bftheta)) = \bfA(\bftheta)^\top( \bfA(\bftheta) \bfw(\bftheta) - \bfc) + \lambda \bfL^\top \bfL \bfw(\bftheta) = 0
	$$
	
	Discussion:
	\begin{itemize}
		\item ignore second term in gradient computation
		\item apply gradient descent/NLCG/BFGS  to minimize $\tilde{J}$
		\item solve least-squares problem in each evaluation of $\tilde{J}$
		\item gradient is only correct if LS problem is solved exactly
	\end{itemize}
	
\end{frame}

\begin{frame}
	\frametitle{Variable Projection for Single Layer}
	
$$
\min_{\bfK,\bfb,\bfW} E(\bfW \sigma(\bfK\bfY +\bfb \bfe_n^\top ), \bfC) + \lambda R(\bftheta,\bfW)
$$
Assume that the regularizer is separable, i.e.,
$$
 R(\bfK,\bfb,\bfW) =   R_1(\bfK,\bfb) +  R_2(\bfW)
$$
and that $R_2$ is convex and smooth. 
\pause
Hence, the projection requires solving the regularized classification problem
$$
\bfW(\bfK,\bfb) = \argmin_{\bfW} E(\bfW\sigma(\bfK\bfY + \bfb \bfe_n^\top), \bfC) + \lambda R_2(\bfW)
$$
practical considerations:
\begin{itemize}
	\item solve for $\bfW(\bfK,\bfb)$ using SVD, Newton (need accuracy)
	\item errors in $\bfW(\bfK,\bfb) \leadsto$ errors in $\nabla \tilde{J}(\bfK)$, $\nabla \tilde{J}(\bfb)$
	\item use gradient-based minimization to solve for $\bfK,\bfb$ 
\end{itemize}
\end{frame}


\begin{frame}
	\frametitle{Practical Considerations in SAA}
	
	Here is a simple but effective SAA-based training algorithm.
	
	\bigskip
	\pause
	
	Pick $(\bfK_0,\bfb_0,\bfW_0)$ randomly and then do one or more steps of:
	\begin{enumerate}
		\item randomly select samples $S$ (large enough)
		\item take a few minimization steps
		\item check and print training error on current batch and validation error
		\item repeat
	\end{enumerate}
	
	\bigskip
	\pause
	
	Possible problems:
	\begin{itemize}
		\item $|S|$ too small $\rightarrow$ training error small but no generalization
		\item $|S|$ too large $\rightarrow$ training too slow
		% \item Too few Newton steps in classification $\rightarrow$ inaccurate gradients, line search fails, \ldots
	\end{itemize}	
\end{frame}
\begin{frame}\frametitle{Discussion: Sample Average Approximation}

Idea: Approximate expected value with samples $S$
$$
	\frac{1}{|S|} \sum_{s\in S} E\left(\bfW \sigma(\bfK\bfy+\bfb),\bfc\right) \approx  {\mathbb{E}}_{(\bfy,\bfc)}\left[E\left(\bfW \sigma(\bfK\bfy+\bfb),\bfc\right)\right]
	$$
	
Advantage: Can use deterministic gradient-based methods, e.g., steepest descent, nonlinear CG, BFGS, Gauss-Newton, VarPro, \ldots

\bigskip
\pause

Drawbacks:
\begin{itemize}
\item
Evaluating gradient needs pass through the entire sample.
\item Sample size must be large enough to avoid overfitting
\end{itemize} 



\end{frame}





\section{Stochastic Approximation}

% \begin{frame}[fragile]
% 	\frametitle{Experiment: Adversarial Example}
%
% 	Suppose you have trained your network $\leadsto \bfK, b, \bfW$ so that validation loss is low. This means that for most examples $\bfy$,
% 	$$
% 		\bfW \sigma(\bfK \bfy + b) \approx \bfc.
% 	$$
%
% 	An adversary might try to fool this classifier by adding a small perturbation $\bfd$ to the example to achieve a desired label $\hat{\bfc}$.
%
% 	\bigskip
%
% 	Formulate as optimization problem
% 	$$
% 		\min_{\bfd} E(\bfW \sigma(\bfK (\bfy + \bfd) + b), \hat{\bfc})
% 	$$
% 	\begin{itemize}
% 		\item setup objective function
% 		\item think about constraints, regularization
% 	\end{itemize}
% \end{frame}


% \end{document}
 




\begin{frame}\frametitle{Stochastic Approximation}
	
	Goal: minimize the expected loss
	$$ {\mathbb{E}}_{(\bfy,\bfc)}\left[E\left(\bfW \sigma(\bfK\bfy+\bfb),\bfc\right)\right]$$
	Assume that each $\bfy_i$, $\bfc_i$ pair is drawn from some (unknown probability distribution). 
	This is a stochastic optimization problem~\cite{bottou2016optimization}. 
	
	\bigskip \pause
	
	Examples: iterations $(\bfK_k, \bfb_k ,\bfW_k) \to (\bfK^*, \bfb^*,\bfW^*)$ that (under certain conditions) decrease the expected value: Stochastic Gradient Descent, ADAM, \ldots
		
	\bigskip \pause
	
		Pro: sample can be small (\emph{mini batch}), often finds global minima for non-convex problems (not much theory though)
	
	\smallskip
		
		Con: how to monitor objective, linesearch, descent, \ldots
\end{frame}


% \begin{frame}[fragile]\frametitle{Review: Supervised Learning Problem}
%
% Most machine learning problems are of the following structure
% $$
% \min_{\bftheta} F(\bftheta,\bfY) + R(\bftheta), \quad \text{ with }\quad  F(\bftheta,\bfY) = \frac1n\sum_{i=1}^n f_i(\bftheta,\bfy_i).
% $$
%
% \bigskip
% \pause
%
% For shallow learning, problem might be convex or have a unique minimum.
% For deep networks, problem is usually not convex and has many local minimum
%
% \end{frame}



\begin{frame}[fragile]\frametitle{Stochastic Gradient Descent}
Consider 
$$
\min_{\bftheta} F(\bftheta,\bfY), \quad \text{ with }\quad  F(\bftheta,\bfY) = \frac1n\sum_{i=1}^n f_i(\bftheta,\bfy_i).
$$

Let ${\cal S}_k \subset \{1,2,\ldots,n\}$. Define the batch objective function as 
$$ F_{{\cal S}_k}(\bftheta) = \frac{1}{|{\cal S}_k|} \sum_{i \in {\cal S}_k} f_i(\bftheta,\bfY_i) $$
Then a straight forward extension  is
$$ \bftheta_{k+1} = \bftheta_k - \mu_k \bfA_k^{-1}  \grad F_{{\cal S}_k}(\bftheta_k) $$


Questions
\begin{itemize}
\item Would the method converge?
\item Under what conditions on $\mu_k,\bfA_k,{\cal S}_k$?
\item How fast?
\end{itemize}

References: original method~\cite{RobbinsMonro1951}, recent surveys~\cite{Bottou2012,Bertsekas2015,bottou2016optimization}
\end{frame}

\begin{frame}[fragile]\frametitle{Stochastic Gradient Descent}

Let ${\cal S}_k \subset \{1,2,\ldots,n\}$. Define the batch objective function as 
$$ F_{{\cal S}_k}(\bftheta) = \frac{1}{|{\cal S}_k|} \sum_{i \in {\cal S}_k} f_i(\bftheta,\bfY_i) $$
Then a straight forward extension   is
$$ \bftheta_{k+1} = \bftheta_k - \mu_k \bfA_k^{-1}  \grad F_{{\cal S}_k}(\bftheta_k) $$

\bigskip
\pause

If $f_i$ are convex, $\bfA_k = \bfI$, $|{\cal S}_k|=1$ and $\mu_k \rightarrow 0$ slowly enough, that is
$$ \sum_{k=1}^{\infty} \mu_k= \infty \quad \text{ and } \quad \sum_{k=1}^{\infty} \mu_k^2 < \infty$$
then SGD converges to stationary point \pause (Ex: $\mu_k = k^{-1}$).

\bigskip
\pause

How fast? Convergence is {\bf sublinear}

\end{frame}


\begin{frame}\frametitle{A Glimpse into the theory}
	Consider the iteration and $\bfA_k=\bfI$
	$$ 
	\bftheta_{k+1} = \bftheta_k - \mu_k  \grad F_{{\cal S}_k}(\bftheta_k) 
	$$
	\pause
	Re-write this as
	$$ 
	\bftheta_{k+1} = \bftheta_k - \underbrace{\mu_k \grad F(\bftheta,\bfY)}_{\rm true\ gradient} -  \underbrace{\mu_k \left ( \grad F_{{\cal S}_k}(\bftheta_k)  - \grad F (\bftheta,\bfY)\right)}_{\rm noise}
	$$
	
	\pause
	
	Note that (unbiased estimator)
	$$
	 {\mathbb{E}} (\grad F_{{\cal S}_k}(\bftheta_k)) = \grad F (\bftheta).
	$$
	
	\pause
	
	Finally note that
	$$
		{\rm Var}\left( \mu_k \grad F_{{\cal S}_k}(\bftheta_k) \right) = \mu_k^2 {\rm Var}\left( \grad F_{{\cal S}_k}(\bftheta_k) \right)
	$$
\end{frame}

\begin{frame}[fragile]\frametitle{Improvements of SGD: Momentum}

Idea: Accelerate convergence by keeping gradient informations from previous batches.


\begin{eqnarray*}
&& \bfS_{k+1} = \gamma \bfS_k  +\mu_k  \grad F_{{\cal S}_k}(\bftheta_k) \\
&& \bftheta_{k+1} = \bftheta_k  - \bfS_{k+1}
\end{eqnarray*}
$\mu_k$ - learning rate, $\gamma$ - momentum 


\bigskip
\pause

Hard to choose in practice, heuristic

$\gamma$ - Start with $0.5$ and increase slowly to 0.9

$\mu$ - problem dependent start small and decrease after a few epoch


\end{frame}

\begin{frame}\frametitle{Improvements of SGD: Nesterov}

	Idea: Predict next iterate using momentum, correct next step using gradient there.
	
	\begin{eqnarray*}
	&& \bftheta_{k+\frac 12} = \bftheta_k  - \gamma \bfS_{k} \\
	&& \bfS_{k+1} = \gamma \bfS_k  + \mu_k  \grad F_{{\cal S}_k}({\bftheta_{k+\frac12}}) \\
	&& \bftheta_{k+1} = \bftheta_k  - \bfS_{k+1}
	\end{eqnarray*}
\end{frame}

\begin{frame}
	\frametitle{Improvements of SGD: AdaGrad}
	
	Idea: Scale step according to size of weights (relation to prior-conditioning in SGD)
	
	\bigskip
	
	
	Iteration:
	\begin{eqnarray*}
	&& \bfD_{k+1} = \bftheta_k^2 + \bfD_k \\
	&& \bfS_{k+1} =  \mu_k {\rm diag}(\bfD_{k+1})^{-1} \grad F_{{\cal S}_k}(\bftheta_{k}) \\
	&& \bftheta_{k+1} = \bftheta_k  - \bfS_{k+1}
	\end{eqnarray*}
	
	
\end{frame}

\begin{frame}[fragile]\frametitle{Discussion: Stochastic Approximation}

General Comments:
\begin{itemize}
\item 
Lots of theory for convex problems
\item
Recall: SGD is not the best tool for most convex problems (see example of least-squares)
\item
Require very careful tuning
\end{itemize}

\bigskip

SGD in deep learning:
\begin{itemize}
	\item currently the main workhorse (DNN $\leadsto$ nonconvex optimization)
	\item why it works? mostly open but some relation to Langevin flow (we also have a few ideas)
	\item observed to regularize problems (theory for quadratic case)
	\item potentially possible to prove global optimality?
\end{itemize}
\end{frame}


% \begin{frame}
% 	\frametitle{Coding: Using SGD for Classification Problem}
%
% 	Outline:
% 	\begin{itemize}
% 		\item Use single layer or ResNet example
% 		\item Change objective function to accept index set $S_k$
% 		\item Use small minibatch
% 		\item Test using peaks example
% 	\end{itemize}
% \end{frame}

\section{Practical Hints}

\begin{frame}[fragile]\frametitle{Practical Hint: Data Preprocessing}

Some practical tips
\begin{itemize}
\item Remove the mean of the data
\item Scale it to be ``reasonable'' scale
\item Data augmentation
\item Some other (domain specific) data transforms (optical flow for motion?)
\end{itemize}

\end{frame}

\begin{frame}[fragile]\frametitle{Regularization for Network Weights}

\begin{itemize}
\item
Note that there are many more degrees of freedom.
\item
Need to add regularization for $\bfK$
\item
$\bfK$ Generally, $\bfK$ is not ``physical'' - difficult to choose reasonable
regularization.
\end{itemize}

\bigskip

The obvious choice: Tikhonov
$$ R(\bfK) = \hf \|\bfK\|^2_F $$
(also called weight decay)


\end{frame}

\begin{frame}[fragile]\frametitle{Learning the weights - Regularization}

\begin{columns}
	\column{.7\textwidth}
More recent, demand that $\bfK$ is sparse
$$ R(\bfK) = \|{\rm vec}(\bfK)\|_1 = \sum_{ij} |\bfK_{ij}| $$

\bigskip

Implementation through soft-thresholding.


After each steepest descent iteration set
$$ \bfK = {\rm softThresh}(\bfK) $$
	
	\column{.3\textwidth}
	\begin{center}
	\includegraphics[width=4cm]{shoftThresh.jpg}
	\end{center}
	
\end{columns}

\vspace{10mm}
\begin{center}
Obtain sparse matrices $\bfK$ that retain only necessary entries
	
\end{center}

\end{frame}




% \begin{frame}[fragile]\frametitle{Coding: Learning the weights }
%
% {\bf Class problem}
%
% \bigskip
%
% Modify your steepest descent, nonlinear CG and SGD codes to work on single layer
% network with soft thresholding.
%
% Test on Circle, peaks, spiral, MNIST and CIFAR10
%
% Compare and report
%
% \end{frame}

\section{Experiments}

\begin{frame}[fragile]
	\frametitle{Test Problems}
	
	Before going to real data, let us try the \emph{inverse crime}. Generate data
	\begin{verbatim}
		n  = 500; nf = 50; nc = 10; m  = 40;
		Wtrue = randn(nc,m);
		Ktrue = randn(m,nf);
		btrue = .1;

		Y     = randn(nf,n);
		Cobs  = exp(Wtrue*singleLayer(Ktrue,btrue,Y));
		Cobs  = Cobs./sum(Cobs,1);		
	\end{verbatim}
	
	\begin{center}
	Goal: Reconstruct \texttt{Wtrue, Ktrue, btrue}!  
	\end{center}
	
	\bigskip
	
	Other cheap test problems: \texttt{PeaksClassification, PeaksRegression, CircleClassification}.
\end{frame}

% \begin{frame}\frametitle{Experiment: Peaks}
%
% 	Compare the three approaches for training a single layer neural network
%
% 	\begin{itemize}
% 		\item \texttt{ESingleLayer\_PeaksSGD.m} - stochastic gradient descent
% 		\item \texttt{ESingleLayer\_PeaksNewtonCG.m} - Newton CG with block-diagonal Hessian approximation
% 		\item \texttt{ESingleLayer\_PeaksVarPro.m} - Fully coupled solver. Eliminate $\bftheta$ and use steepest descent/Newton CG for reduced problem.
% 	\end{itemize}
%
% \end{frame}

\section{Summary}
\begin{frame}
	\frametitle{$\Sigma$ : Single-Layer Neural Networks }
	$$ \min_{\bfK,\bfW,\bfb} E(\bfW\sigma(\bfK \bfY + \bfb\bfe_n^\top), \bfC_{\rm obs}) + \lambda R(\bfW,\bfK,\bfb)$$
	
	\begin{itemize}
		\item transform data, increase dimension $\leadsto$ approximation power
		\item Extreme Learning Machines: random nonlinear feature extractor
		\item More common to train $\bfK, \bfb, \bfW$
		\item Training problem is non-convex and stochastic
		\item SAA methods: Pick large sample and use deterministic tools (easy to parallelize, fast convergence if done right, but can be trapped in local minima)
		\item SA methods: small sample and random steps (easy to code, difficult to parallelize, need to choose hyper parameter)
	\end{itemize}
\end{frame}



\begin{frame}[allowframebreaks]
	\frametitle{References}
\bibliographystyle{abbrv}
\bibliography{NumDNN}

\end{frame}

\end{document}

