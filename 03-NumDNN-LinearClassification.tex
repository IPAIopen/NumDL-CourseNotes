\documentclass[12pt,fleqn,handout]{beamer}


\xdefinecolor{lavendar}{rgb}{0.8,0.6,1}
\xdefinecolor{olive}{cmyk}{0.64,0,0.95,0.4}
%\xdefinecolor{olive}{cmyk}{1,0,0,0}
\xdefinecolor{mag}{cmyk}{0.1,1,0,0.2}
\xdefinecolor{lblue}{rgb}{0,0,1.5}
\xdefinecolor{lred}{rgb}{1,0,0}
\xdefinecolor{mine}{cmyk}{1,0,0.2,0}
\xdefinecolor{bluel}{cmyk}{0.1,0,0.9,0.4}

\usepackage{amsmath,amssymb,dsfont,mathrsfs}
\usepackage{tikz,pgflibraryplotmarks}
\usepackage{multimedia}
\usepackage{wasysym}
\usepackage{rotating}
\usepackage{algorithm,algorithmic}
\usepackage{graphicx} % more modern
\usepackage{subfigure}
\usepackage{booktabs}

\usepackage{pgfplots}
\usepackage{verbatim}

\usepackage{setspace}
\newlength\iwidth
\newlength\iheight

\newcommand\makebeamertitle{\frame{\maketitle}}%
\graphicspath{{./images/}}
\setbeamertemplate{navigation symbols}{}
\addtobeamertemplate{navigation symbols}{}{%
    \usebeamerfont{footline}%
    \usebeamercolor[fg]{footline}%
	\insertshorttitle
    \;--
    \insertframenumber
}

\newcommand{\sectionstart}{
	\only<beamer>{
 	\begin{frame}% (fold)
 		\begin{centering}\Huge \insertsection \par\end{centering}
 	\end{frame}% frame the_application (end)
	}
 }


% make bibliography entries smaller
\usepackage{natbib}
\setbeamertemplate{bibliography item}{[\theenumiv]}
\renewcommand\bibfont{\scriptsize}
\setbeamertemplate{frametitle continuation}[from second]
\newcommand{\tcr}{\textcolor{red}}
\newcommand{\tcrd}{\textcolor{red}}
\newcommand{\tcb}{\textcolor{bluel}}
\newcommand{\tcm}{\textcolor{mag}}
\newcommand{\tcg}{\textcolor{olive}}

\newcommand{\R}{\mathbb{R}}
\newcommand{\C}{\mathbb{C}}

% bold lower-case for vectors
\newcommand{\bfa}{{\bf a}}
\newcommand{\bfb}{{\bf b}}
\newcommand{\bfc}{{\bf c}}
\newcommand{\bfs}{{\bf s}}
\newcommand{\bfm}{{\bf m}}
\newcommand{\bfd}{{\bf d}}
\newcommand{\bfe}{{\bf e}}
\newcommand{\bfu}{{\bf u}}
\newcommand{\bfy}{{\bf y}}
\newcommand{\bfx}{{\bf x}}
\newcommand{\bfh}{{\bf h}}
\newcommand{\bfw}{{\bf w}}
\newcommand{\bfv}{{\bf v}}
\newcommand{\bfr}{{\bf r}}
\newcommand{\bfz}{{\bf z}}
\newcommand{\bff}{{\bf f}}
\newcommand{\bfp}{{\bf p}}


% bold upper-case for linear operators
\newcommand{\bfA}{{\bf A}}
\newcommand{\bfB}{{\bf B}}
\newcommand{\bfZ}{{\bf Z}}
\newcommand{\bfM}{{\bf M}}
\newcommand{\bfC}{{\bf C}}
\newcommand{\bfD}{{\bf D}}
\newcommand{\bfQ}{{\bf Q}}
\newcommand{\bfJ}{{\bf J}}
\newcommand{\bfG}{{\bf G}}
\newcommand{\bfI}{{\bf I}}
\newcommand{\bfP}{{\bf P}}
\newcommand{\bfK}{{\bf K}}
\newcommand{\bfY}{{\bf Y}}
\newcommand{\bfW}{{\bf W}}
\newcommand{\bfR}{{\bf R}}
\newcommand{\bfL}{{\bf L}}
\newcommand{\bfF}{{\bf F}}
\newcommand{\bfT}{{\bf T}}
\newcommand{\bfS}{{\bf S}}
\newcommand{\bfX}{{\bf X}}
\newcommand{\bfU}{{\bf U}}
\newcommand{\bfV}{{\bf V}}
\newcommand{\bfH}{{\bf H}}

\newcommand{\diag}{\rm diag}

\newcommand{\calF}{\mathcal{F}}
\newcommand{\calN}{\mathcal{N}}
\newcommand{\calT}{\mathcal{T}}
\newcommand{\calX}{\mathcal{X}}




\newcommand{\hf}{{\frac 12}}
\newcommand{\bftheta}{{\boldsymbol \theta}}
\newcommand{\bfmu}{{\boldsymbol \mu}}
\newcommand{\bfxi}{{\boldsymbol \xi}}

\newcommand{\bfLambda}{{\boldsymbol \Lambda}}
\newcommand{\bflambda}{{\boldsymbol \lambda}}
\newcommand{\bfSigma}{{\boldsymbol \Sigma}}
\newcommand{\bfepsilon}{{\boldsymbol \epsilon}}

\newcommand{\E}{\vec E}
\newcommand{\B}{\vec B}

\newcommand{\vu}{  {\vec {\bf u}}}

\newcommand{\grad}{  {\vec {\bf \nabla}}}

\newcommand{\lfrownie}{\textcolor{red}{\large{\frownie}}}
\newcommand{\lsmiley}{\textcolor{green}{\large{\smiley}}}

\newcommand{\curl}{\ensuremath{\nabla\times\,}}
\renewcommand{\div}{\nabla\cdot\,}
\newcommand{\divh}{\nabla_h\cdot\,}
\renewcommand{\grad}{\ensuremath{\nabla}}

\DeclareMathOperator*{\argmin}{arg\,min}


\title{Linear Models for Classification}
\subtitle{Numerical Methods for Deep Learning}
\date{}
\begin{document}

\makebeamertitle

\begin{frame}
	\frametitle{Learning Objective: Linear  Classifiers}
	
	In this module we review computational methods for linear classification.
	
	\bigskip
	
	Learning tasks:
	\begin{itemize}
		\item binary classification $\leadsto$ logistic regression
		\item multiclass problems $\leadsto$ multinomial logistic regression
	\end{itemize}
	
	\bigskip
	
	Numerical methods:
	\begin{itemize}
		\item convex optimization
		\item steepest descent
		\item Newton's method
	\end{itemize}
\end{frame}


\section{Logistic Regression and Softmax} % (fold)
\label{sec:logistic_regression_and_softmax}

\begin{frame}
	\frametitle{Logistic Regression}
	
	
	Assume our data falls into two classes. Denote by $\bfc_{\rm obs}(\bfy)$ the probability that example $\bfy\in\R^{n_f}$ belongs to first category.
	
	\bigskip
	
	Since output of our classifier $f(\bfy,\bftheta)$ is supposed to be a probability, use logistic sigmoid
	$$
		\bfc(\bfy,\bftheta) = \frac{1}{1 + \exp\left(-f(\bfy,\bftheta)\right)}.
	$$
	
	\bigskip
	
	Example (Linear Classification): If $f(\bfy,\bftheta)$ is a linear function (adding bias is easy), $\bftheta = \bfw \in \R^{n_f}$ and
	$$
		\bfc(\bfy,\bfw) = \frac{1}{1 + \exp(-\bfw^\top\bfy)}.
	$$
	
	\begin{center}
		this module: consider linear models and focus on loss
	\end{center}
	
\end{frame}

\begin{frame}
	\frametitle{Multinomial Logistic Regression}
	
	Suppose data falls into $n_c\geq 2$ categories and the components of $\bfc_{\rm obs}(\bfy) \in \Delta_{n_c}$ contain probabilities for each class. 
	
	\bigskip
	
	$\Delta_{n_c}$ is the unit simplex in $\R^{n_c}$
	\begin{equation*}
		\Delta_{n_c} = \{ \bfc \in [0,1]^{n_c} \ : \ \bfe_{n_c}^\top \bfc = 1\}.
	\end{equation*}
	
	\bigskip
	 
	Applying the logistic sigmoid to each component of $f(\bfy,\bfW)$ not enough to be in $\Delta_{n_c}$. Use
	$$
		\bfc(\bfy,\bfW) = \left( \frac{1}{\bfe_{n_c}^\top\exp(\bfW\bfy)} \right) \; \exp(\bfW\bfy).
	$$
	
	\bigskip
	
	Note: Division and $\exp$ are applied element-wise!
	
\end{frame}
\begin{frame}
	\frametitle{Logistic Regression - Loss Function}
	
	How similar are $\bfc(\cdot,\bfW)$ and $\bfc_{\rm obs}(\cdot)$?

	\begin{columns}
		\column{.7\textwidth}
	Naive idea: Let $\bfY \in \R^{n_f\times n}$ be examples with class probabilities $\bfC_{\rm obs} \in [0,1]^{n_c\times n}$, use 
	$$
	\phi(\bfW)= \frac{1}{2n} \sum_{j=1}^n\|\bfc(\bfy_j,\bfW) - \bfc_{j,{\rm obs}} \|_F^2
	$$

Problems
\begin{itemize}
	\item ignores that $\bfc(\cdot,\bfW)$ and $\bfc_{\rm obs}(\cdot)$ are distributions.
	\item leads to non-convex objective function
\end{itemize} 
		
		\column{.4\textwidth}
		
		\begin{tabular}{c}
			Frobenius\\
			\includegraphics[width=\textwidth]{Class_Frobenius}\\
			Cross Entropy\\
			\includegraphics[width=\textwidth]{Class_CrossEntropy}
		\end{tabular}
	\end{columns}
	
\end{frame}


\begin{frame}
	\frametitle{Example: Designing a Code}
	
	Goal: Communicate using minimal number of bits.
	
	\bigskip
	\pause
	
	\invisible<beamer|1>{
	Example: Bob talks $\bfc = [1/2,1/4,1/8,1/8]$ of the time about dogs, cats, fish, and birds, respectively.
	
	How many bits need to be transferred on average?}

	\bigskip
	
	\invisible<beamer|-2>{
	\begin{center}
		\begin{tabular}{l|cc}
			word & naive code & \invisible<beamer|-3>{better code} \\ \hline
			dog  & 00         & \invisible<beamer|-3>{ 0 }         \\ 
			cat  & 01         & \invisible<beamer|-3>{ 10   }      \\
			fish & 10         & \invisible<beamer|-3>{ 110  }      \\ 
			bird & 11         & \invisible<beamer|-3>{ 111   }     \\
		\end{tabular}		
	\end{center}
	}
	
	\invisible<beamer|-4>{
	\begin{center}
		Idea: Quantify information content in probability distribution using average length.		
	\end{center}
	}
	
	
	\only<beamer|5>{}
	

\end{frame}

\begin{frame}
	\frametitle{Entropy}

	Note: Length of word depends on its probability being used. How long should a word be?
	
	\bigskip
	\pause
	
	Optimal choice for information for any category
$$ I = \log_2(\bfc_j^{-1}) = -\log_2(\bfc_j) $$


\bigskip

The larger $\bfc_j$, the more frequent we use it, the shorter the word should be. 

\bigskip
\pause

The entropy is the average (expectation) of information over all classes.

$$ E(\bfc) = -\sum \bfc_j \log_2(\bfc_j) = -\bfc^{\top} \log_2(\bfc) $$

\end{frame}

\begin{frame}
	\frametitle{Example: Designing a Code - 2}

	Entropy for Bob's code is
	$$
		\frac{1}{2} \log\left(2\right) + \frac{1}{4} \log\left(4\right) + 2 \frac{1}{8} \log\left(8\right) = 1.75
	$$
	average length of word is 1.75 bits $<$ 2 bits for naive code!
	
\bigskip


\bigskip
\pause

For the complete tutorial on entropy, read
{\small
\tt{http://colah.github.io/posts/2015-09-Visual-Information/}
}

\end{frame}

\begin{frame}
	\frametitle{Properties of Entropy}

\begin{itemize}
\item recall $\lim_{x\rightarrow 0} x \log x = 0$
\item prefer sparse distributions (why?)
\item has been used in compressed sensing type methods
\end{itemize}

\begin{center}
\begin{figure}
\includegraphics[width=5cm]{entropy.jpg}
\caption{The entropy of a vector $\bfc = (c_1, c_2)^\top$}
\end{figure}
\end{center}


\end{frame}

\begin{frame}
	\frametitle{Cross Entropy}


Measure the average word length when using code designed for $\bfc$ for sending information with probability $\widehat \bfc$
$$ E(\widehat\bfc, \bfc) = - {\widehat \bfc}^{\top} \log(\bfc). $$

\bigskip

Clearly
$$ E(\widehat\bfc,  \bfc) \ge E(\bfc, \bfc) $$

\bigskip
\pause

Example: Alice talks $\bfc = [1/8,1/2,1/4,1/8]$ of the time about dogs, cats, fish, and birds, respectively.
If she used Bob's code, the average word length would be 
$$
	\frac{1}{8} \log\left(2\right) + \frac{1}{2} \log\left(4\right) + \frac{1}{4} \log\left(8\right) + \frac{1}{8} \log\left(8\right) = 2.25 > 1.75
$$

\bigskip
\pause

$E$ measures how similar the distributions $\bfc$ and $\widehat \bfc$ are.


\bigskip
\pause

One flaw: $ E(\bfc, \widehat \bfc) \not= E(\widehat \bfc, \bfc)$ (verify for our example!)


\end{frame}

\begin{frame}
	\frametitle{Cross Entropy for Logistic Regression - 1}

Recall: For a single example and two classes we have
 $$
  \bfc(\bfy,\bfw)  = \begin{pmatrix}
  	{\frac 1 {1+\exp(-\bfw^{\top} \bfy)}} \\
	1-{\frac 1 {1+\exp(-\bfw^{\top} \bfy)}}
  \end{pmatrix} 
  \stackrel{\rm def}{=} 
  \begin{pmatrix}
  	h(\bfw^{\top}\bfy) \\1-h(\bfw^{\top}\bfy)
  \end{pmatrix}
 $$

Assume we have the observation $\bfc_{\rm obs} = \begin{pmatrix}
	\bfc_{\rm obs} \\ 1-\bfc_{\rm obs}
\end{pmatrix}$ then
\begin{align*}
 E(\bfc_{\rm obs}, \bfc) & = -\bfc_{\rm obs}^\top\log(\bfc(\bfy,\bfw)) \\
 & = -\bfc_{\rm obs} \log(h(\bfw^{\top}\bfy)) - (1-\bfc_{\rm obs})
\log(1-h(\bfw^{\top}\bfy)). 
\end{align*}
where
$$
 h(z) = \frac{1}{1+\exp(-z)}
$$
\end{frame}

\begin{frame}
	\frametitle{Cross Entropy for Logistic Regression - 2}


In the case we have many examples need to sum over the data
\begin{align*}
	\bfC(\bfY,\bfw) & = \begin{pmatrix}
		{\frac 1 {1+\exp(-\bfw^\top \bfY)}} \\
		1-{\frac 1 {1+\exp(\bfw^\top\bfY)}}
	\end{pmatrix}  \stackrel{def}{=} \begin{pmatrix}
		h(\bfw^\top \bfY)\\ 1-h(\bfw^\top\bfY)
	\end{pmatrix}
	\in \R^{2\times n}
\end{align*}
Assume we have the observation $\bfc_{\rm obs}\in\R^n$. Define
$$
\bfC_{\rm obs} = \begin{pmatrix}
	\bfc_{\rm obs}^\top \\ 1-\bfc_{\rm obs}^\top
\end{pmatrix} \in \R^{2 \times n}.
$$

Then the cross entropy is
\begin{align*}
	E(\bfC_{\rm obs}, \bfC)  =& - \frac{1}{n} {\rm tr}(\bfC_{\rm obs}^\top \bfC) \\
	 						 = & -\frac{1}{n} \bfc_{\rm obs}^{\top} \log(h(\bfw^\top\bfY) \\
							   & - \frac{1}{n}(1-\bfc_{\rm obs})^{\top}\log(1-h(\bfw^\top\bfY)))
\end{align*}

\end{frame}

\begin{frame}
	\frametitle{Cross Entropy for Multinomial Logistic Regression}
	
	Similarly, for general case ($n_c\geq2$ classes, $n$ examples). 
	
	Recall: 
	$$\bfC(\bfY,\bfW)  =   \exp(\bfW \bfY)\; {\rm diag} \left({\frac 1 {\bfe_{n_c}^\top\exp(\bfW \bfY)}}  \right)  $$

\bigskip

Get cross entropy by summing over all examples

$$ E(\bfC_{\rm obs},\bfC(\bfY,\bfW)) = -\frac{1}{n}{\rm tr}( \bfC_{\rm obs}^\top\log(\bfC(\bfY,\bfW)) ). $$

We will also call this the \emph{softmax} (cross-entropy) function.


\end{frame}


\begin{frame}\frametitle{Simplifying the Softmax Function}

Let $\bfS = \bfW \bfY$, then 
$$ 
E(\bfC_{\rm obs},\bfS) = -\frac{1}{n}{\rm tr}\left(\bfC_{\rm obs}^{\top} \log\left({\exp(\bfS)}{\rm diag}\left(\frac{1}{\bfe_{n_c}^\top \exp(\bfS) }\right) \right)\right). 
$$
\pause

Verify that this is equal to
\begin{align*}
	 E(\bfC_{\rm obs},\bfS) = & -\frac{1}{n}\bfe_{n_c}^{\top} \left( \bfC_{\rm obs} \odot \bfS\right) \bfe_{n} \\
	 & + \frac{1}{n} \bfe_{n_c}^\top \bfC_{\rm obs}\log\left(\bfe_{n_c}^\top\exp(\bfS)\right)^\top
\end{align*}
($\odot$ is Hadamard product, $\exp$ and $\log$ component-wise)
\bigskip
\pause


If $\bfC_{\rm obs}$ has a unit row sum (why?) then $ \bfe_{n_c}^\top \bfC_{\rm obs}^\top = \bfe_n^\top$
and 
$$ E(\bfC_{\rm obs},\bfS) = -\frac{1}{n}\bfe_{n_c}^{\top} \left( \bfC_{\rm obs} \odot \bfS\right) \bfe_{n} 
+ \frac{1}{n}\log(\bfe_{n_c}^\top\exp(\bfS))\bfe_{n} $$


\end{frame}



\begin{frame}\frametitle{Numerical Considerations}

Scale to prevent overflow. Note that for an arbitrary $s\in\R$ we have
$$ E(\bfC_{\rm obs},\bfW\bfY-s) = E(\bfC_{\rm obs},\bfW\bfY) $$

This prevents overflow, but may lead to underflow (and divisions by zero).

\bigskip
\pause

Note that $s$ does not need to be the same in each row (example). Hence, 
we can choose $\bfs = \max(\bfW\bfY,[],1 ) \in \R^{1\times n}$ to avoid underflow and overflow.

\bigskip
\begin{center}
For stability use $E(\bfC_{\rm obs},\bfS)$ where $ \bfS = \bfW \bfY - \bfe_{n_c} \bfs$.
\end{center}
\end{frame}

\begin{frame}[fragile]\frametitle{Test Problem: Linear Classification}

Generate data that is linearly separable:

\begin{center}
	\includegraphics[height=40mm]{linearClass}
\end{center}
\vspace*{-5mm}
\begin{verbatim}

a = 3; b = 2;

Y = randn(2,500);
C = a*Y(1,:) + b*Y(2,:) + 1;
C(C>0) = 1; C(C<0) = 0;
C = [C; 1-C]

\end{verbatim}


\end{frame}


\begin{frame}[fragile]\frametitle{Coding: Softmax Regression Objective Function}


Write a function that computes the softmax function given a data matrix $\bfY$,
its class $\bfC$, and a matrix $\bfW$.

\bigskip

\begin{verbatim}
function[E] = softmaxFun(W,Y,C)
% Your code here
end
\end{verbatim}
Here are some tests your code should master
\begin{itemize}
	\item result should always be scalar
	\item result should not depend on the ordering of the columns of $\bfY$
	\item for a given $(\bfy,\bfc)$ the output of $E(\bfy,\bfc)$ should be the same as $E(\bfy\bfe^\top, \bfc\bfe^\top)$.
	\item $\bfW = \alpha \bfI$ and $\bfY = \bfC$ not cause overflow for any $\alpha$
\end{itemize}
\end{frame}


\begin{frame}\frametitle{Linear Classification}

If $\bfW$ can separate the classes then the goal is to minimize the cross entropy (with some potential regularization)

\begin{align*}
 \bfW^* = & \argmin_{\bfW} \quad - \frac{1}{n}\bfe_{n_c}^{\top} \left( \bfC_{\rm obs} \odot \bfS\right) \bfe_{n} 
+ \frac{1}{n} \log(\bfe_{n_c}^\top \exp(\bfS))\bfe_n \\
          & \text{subject to} \;\bfS = \bfW \bfY - \bfe_{n_c} \bfs
\end{align*}
\pause

This is a smooth convex optimization problem \\ $\Rightarrow$ many existing optimization techniques will work

\bigskip

For large-scale problems, use derivative-based optimization algorithm. (Examples: Steepest Descent, Newton-like methods,  Stochastic Gradient Descent, ADMM, \ldots)

\bigskip

References: \cite{NocedalWright2006,BoydVandenberghe2004,Beck2014,WuFungEtAl2019SoftmaxADMM}
\end{frame}

 \begin{frame}\frametitle{Differentiating the Softmax Function - 1}

We need to compute the derivative of the cross entropy function with respect to $\bfW$.
Three hints:
\begin{itemize}
\item $\sum \bfw \odot \bfy = \bfw^{\top} \bfy $
\item $\grad_{\bfw} (\bfw^{\top} \bfy) = \bfy$
\item ${\rm vec} (\bfW \bfY) = (\bfY^\top \otimes \bfI) {\rm vec} (\bfW) = (\bfI \otimes \bfW) {\rm vec}(\bfY)$
\end{itemize}
where $\otimes$ is the Kronecker product.

\bigskip



We use the common notation:
\begin{itemize}
	\item ${\rm vec}(\bfA)$ reshapes matrix $\bfA$ (column-wise) into vector
	\item ${\rm mat}(\bfa)$ reshapes vector $\bfa$ into matrix, ${\rm mat}({\rm vec}(\bfA)) = \bfA$.
	\item ${\rm diag}(\bfA) = {\rm diag}({\rm vec}( \bfA))$ builds diagonal matrix with entries given by $\bfA$. 
\end{itemize}


\end{frame}

\begin{frame}[fragile]\frametitle{Differentiating the Softmax Function - 2}

Do it in three steps:
\begin{enumerate}
	\item $\nabla_{\bfS} E(\bfS)$ with $\bfS = \bfY \bfW$ (assume w.l.o.g. no shift)
	\item $\nabla_{\bfW} \bfS = \nabla_{\bfW} (\bfW \bfY)$
	\item use chain rule to get $\nabla_{\bfW} E(\bfW\bfY)$.
\end{enumerate}

\bigskip
\pause

Break the first step down into two terms
$$ E(\bfS) = \frac{1}{n} \overbrace{-{\rm tr}(\bfC_{\rm obs}^{\top} \bfS)}^{E1} + \frac{1}{n} \overbrace{\log(\bfe_{n_c}^\top\exp(\bfS))\bfe_n}^{E2}, $$


\bigskip
\pause

First term is linear

$$\grad_{\bfS}E_1 =  \grad_{\bfS} {\rm tr}(\bfC_{\rm obs}^{\top} \bfS) = \bfC_{\rm obs}. $$

\end{frame}

\begin{frame}[fragile]\frametitle{Differentiating the Softmax Function - 3}


$$ E(\bfS) = \frac{1}{n}\overbrace{-{\rm tr}(\bfC_{\rm obs}^{\top} \bfS)}^{E1} + \frac{1}{n} \overbrace{\log(\bfe_{n_c}^\top\exp(\bfS))\bfe_n}^{E2}  $$


\bigskip

Second  term requires a bit more care

$$ \bfJ_{\bfS}E_2 = \bfJ_{\bfS} \bfe_n^{\top}\log(\bfe_{n_c}^\top \exp(\bfS)) = 
 \bfe_n^{\top}\bfJ_{\bfS}  \log(\bfe_{n_c}^\top\exp(\bfS))  $$

\pause 
 and
 
$$ \bfJ_{\bfS}  \log(\bfe_{n_c}^\top\exp(\bfS)) = {\rm diag}
\left( {\frac {1}{\bfe_{n_c}^\top\exp(\bfS)}}\right)
\bfJ_{\bfS}\left(\bfe_{n_c}^\top\exp(\bfS) \right) $$
 

\end{frame}


\begin{frame}[fragile]\frametitle{Differentiating the Softmax Function - 4}
	Recall:
$$ \bfJ_{\bfS}  E_2 = \bfe_n^\top  {\rm diag}
\left( {\frac {1}{\bfe_{n_c}^\top\exp(\bfS)}}\right)
\bfJ_{\bfS}\left(\bfe_{n_c}^\top\exp(\bfS) \right) $$

Focus on last term:
\begin{align*}
   \bfJ_{\bfS} (\bfe_{n_c}^\top \exp(\bfS)) & = \bfJ_{\bfS} \left(\left(\bfI \otimes \bfe_{n_c}^{\top} \right)  {\rm vec}(\exp(\bfS))\right)\\
   &  = (\bfI \otimes \bfe_{n_c}^{\top})  {\rm diag}( {\rm vec}(\exp(\bfS)))
\end{align*}
 
 \pause
 Putting it together
 
 
 $$ \bfJ_{\bfS}E_2 =  \bfe_n^{\top}\ {\rm diag}
\left( {\frac {1}{\bfe_{n_c}^\top \exp(\bfS)}}\right) \  ( \bfI \otimes \bfe_{n_c}^{\top})  {\rm diag}( {\rm vec}(\exp(\bfS))) $$


\pause

\begin{center}
	Left to do: Take transpose
\end{center}


\end{frame}

\begin{frame}[fragile]\frametitle{Differentiating the Softmax Function - 5}

 
 $$ \grad_{\bfS}E_2 =     {\rm diag}( {\rm vec}(\exp(\bfS))) (\bfI \otimes \bfe_{n_c})
  {\rm diag}
\left( {\frac {1}{\bfe_{n_c}^\top\exp(\bfS)}}\right) \bfe_n
     $$
\pause
simplifying

 $$ \grad_{\bfS}E_2 =     {\rm diag}( {\rm vec}(\exp(\bfS))) (\bfI \otimes \bfe_{n_c})
\left( {\frac {1}{\bfe_{n_c}^\top\exp(\bfS)}}\right)
     $$
 
\pause    
     Avoid ${\rm diag}$ and simplify using  matrix representation
     
 $$ \grad_{\bfS}E_2 =     \exp(\bfS) \odot 
\left( \bfe_{n_c} \left({\frac {1}{\bfe_{n_c}^\top \exp(\bfS)}}\right)\right) 
     $$
     
     \bigskip
     \pause
     Finally combine gradients of $E_1$ and $E_2$
     
$$     \grad_{\bfS}E =  -\frac{1}{n}\bfC_{\rm obs} + \frac{1}{n}\exp(\bfS) \odot 
\left( \bfe_{n_c} \left({\frac {1}{\bfe_{n_c}^\top \exp(\bfS)}}\right)\right) . $$
     
    

\end{frame}

\begin{frame}[fragile]\frametitle{Differentiating the Softmax Function - 6}

$$ E(\bfW) = -\frac{1}{n}\overbrace{{\rm tr}(\bfC_{\rm obs}^{\top} (\bfW\bfY))}^{E1} + \frac{1}{n} \overbrace{\bfe_n^{\top}\log(\bfe_{n_c}^\top \exp(\bfW\bfY))}^{E2}  $$

\bigskip
Note that
$$
	\nabla_{\bfW} \bfS = \nabla_{\bfW} (\bfW \bfY) = \nabla_{\bfW} \left( (\bfY^\top \otimes \bfI) {\rm vec}(\bfW) \right) = \bfY \otimes \bfI. 
$$


\bigskip

Hence, applying the chain rule gives
     
$$     \grad_{\bfW}E = \frac{1}{n} \left( -\bfC_{\rm obs} + \exp(\bfS) \odot 
\left( \bfe_{n_c} \left({\frac {1}{\bfe_{n_c}^\top \exp(\bfS)}}\right)\right) \right)\bfY^\top. $$
     

\end{frame}


\begin{frame}[fragile]\frametitle{Coding: Differentiating the Softmax Function}

Extend your softmax function, so that it returns the gradient if needed.

\begin{verbatim}
function[E,dE] = softmaxFun(W,Y,C)

% Your code from before

if nargout > 1

% Your code for gradient here
end

end
\end{verbatim}

\end{frame}



\begin{frame}[fragile]\frametitle{Testing your Derivatives}

Your derivatives are assumed to be wrong unless you prove otherwise.

Test based on Taylor theorem. Let $\bfW$ be fixed and $\bfD$ be a random direction (same size as $\bfW$):

{\begin{small}
\begin{center}
\begin{tabular}{c|c|c}
 $h$  & $E(\bfW + h\bfD) -    E(\bfW)$ &  $E(\bfW + h\bfD) -    E(\bfW) - h {\rm tr}(\bfD^{\top} \nabla E(\bfW))$ \\
 \hline
 1   &   &   \\
 $2^{-1}$   &   &   \\
 $2^{-2}$   &   &   \\
 $2^{-3}$   &   &   \\
 $2^{-4}$   &   &   \\
 $2^{-5}$   &   &
 \end{tabular}
 \end{center}
 \end{small}}


\pause

First column should decay as ${\cal O}(h)$ \\
Second column should decay as ${\cal O}(h^2)$ \\


\end{frame}



\begin{frame}\frametitle{Derivative-Based Optimization: Steepest Descent}


To minimize the energy go ``down-hill''
\begin{center}
	\includegraphics[width=5cm]{steepestDescent.jpg}
\end{center}

Iterate:
$$ \bfW_{k+1} = \bfW_k + \alpha \bfD, \quad \bfD =  -\grad E(\bfW_k). $$
Guaranteed to be a descent direction but need to make sure that
$$ E(\bfW_k + \alpha \bfD) < E(\bfW_k)  $$

\end{frame}

\begin{frame}
	\frametitle{Line Search Problem}
	\begin{center}
		\iwidth=40mm
		\iheight=30mm
		\begin{tabular}{cc}
			objective function
			&
			linesearch
			\\
			\scriptsize
			\input{images/lineSearch-contour.tex}
			&
			\scriptsize
			\input{images/lineSearch-linesearch.tex}
		\end{tabular}
	\end{center}
	Let $E$ be the cross entropy, $\bfW_k$ the current weights, and $\bfD$ the search direction.  The line search problem is:
	\begin{equation*}
		\min_{\alpha > 0} \phi(\alpha) \quad \text{ where } \quad \phi(\alpha) = E(\bfW_k + \alpha \bfD).
	\end{equation*}
\end{frame}


\begin{frame}\frametitle{Armijo Line Search}

A method for inexact line search


\begin{itemize}
\item Start with $\alpha = \alpha_0$
\item Test $ E(\bfW + \alpha \bfD) {<} E(\bfW)$
\item If fail $\alpha \leftarrow \hf \alpha$
\end{itemize}

\bigskip
\pause

A few (small) but helpful tricks
\begin{itemize}
\item Choose your $\alpha_0$ based on the problem
\item If line search is needed ($\alpha\not=\alpha_0$) at iteration $k$ set $\alpha_0 = \alpha_k$ in next iteration.
\item If no line search is needed ($\alpha =\alpha_0$) at iteration $k$ set $\alpha_0 = \gamma\alpha_k$, $\gamma>1$ in next iteration.
\end{itemize}


\end{frame}


\begin{frame}[fragile]\frametitle{Coding: Steepest Descent }

Write a code for steepest descent with Armijo linesearch.

\begin{verbatim}
function W = steepestDescent(E,W,param)
% Inputs:
%     E  - function that provides value and gradient
%     W  - starting guess
%  param - struct with parameters

alpha     = param.maxStep; % max step size
maxIter   = param.maxIter; % max number of iterations

for i=1:maxIter

% Your code here
end
\end{verbatim}

\end{frame}


\section{Intro to Newton's Method} % (fold)
\label{sec:logistic_regression_and_softmax}

\begin{frame}\frametitle{Newton-like Methods}

Goal: Solve $\min_{\bfW} E(\bfW)$. Consider $k$th iteration. Assume $E$ convex.

\bigskip
\pause

To find optimal step $\bfD$, use Taylor's theorem
{\small{
$$ E(\bfW_k + \bfD) = E(\bfW_k) + \grad E(\bfW_k)^{\top} \bfD +
\hf \bfD^{\top} \grad^2 E(\bfW_k) \bfD +
{\cal O}(\|\bfD\|^3)  $$}}
and differentiate w.r.t $\bfD$ 
\pause
to obtain
$$ \grad^2 E(\bfW_k) \bfD = -\grad E(\bfW_k). $$

\pause

Practical Newton methods (see, e.g., \cite[Ch.7]{NocedalWright2006})
\begin{itemize}
\item do not compute $\bfD$ accurately (add line search for safety)
\item use, e.g., Conjugate Gradient (CG) methods
\item do not generate $\grad^2 E$ since CG only needs mat-vecs
\item give quadratic/superlinear/good linear convergence
\end{itemize}

\end{frame}

\begin{frame}[fragile]\frametitle{Newton-like Methods for Softmax}

Need to compute Hessian $\nabla^2 E$. Recall: 
\begin{align*}
	\grad_{\bfW}E &= \frac{1}{n} \left( -\bfC_{\rm obs} + \exp(\bfS) \odot 
	\left( \bfe_{n_c} \left({\frac {1}{\bfe_{n_c}^\top \exp(\bfS)}}\right)\right) \right)\bfY^\top\\
              & =  \grad_\bfS E(\bfS) \bfY^\top,
\end{align*}
where $\bfS = \bfW \bfY$. 
\pause
For $\bfw = {\rm vec}(\bfW)$ and $\bfs = {\rm vec}(\bfS)$ we know that
$$
\nabla^2_\bfw E(\bfw) = (\bfY \otimes \bfI_{n_c}) \; \nabla_{\bfs}^2 E(\bfs)\;  (\bfY^\top \otimes \bfI_{n_c})
$$

\pause

Remarks:
\begin{itemize}
\item size of $\nabla^2_\bfs E$ is $n_c n \times  n_c n$, typically sparse
\item size of $\nabla^2_\bfw E$ is $n_c n_f \times n_c n_f$, typically dense
\item building Hessian can be costly (when $n$ is large)
\item Hessian is spd since $E$ is convex in $\bfS$
\end{itemize}


\end{frame}

\begin{frame}[fragile]\frametitle{Hessian of Softmax Function - 1}

Recall
$$ \grad_\bfS E = \frac{1}{n} \left(-\bfC_{\rm obs} + \exp(\bfS)  \odot \frac 1 {\bfe_{n_c}\bfe_{n_c}^\top\exp(\bfS)} \right) $$

Let's first vectorize this  $\bfs = {\rm vec}(\bfS)$ and $\bfc = {\rm vec}(\bfC_{\rm obs})$

$$ \grad_\bfs E = \frac{1}{n} \left(- \bfc + \exp(\bfs)  \odot \frac{1 }{(\bfI \otimes (\bfe_{n_c} \bfe_{n_c}^{\top}))\exp(\bfs)} \right) $$

\bigskip
\pause

Use product rule

\begin{eqnarray*}
\nabla^2_\bfs E &=& \frac{1}{n}{\rm diag} \left( {\frac 1 {(\bfI \otimes (\bfe_{n_c} \bfe_{n_c}^{\top})) \exp(\bfs)}} \right)
\bfJ_\bfs  \exp(\bfs)   +  \\
&&
\frac{1}{n}{\rm diag} (\exp (\bfs)) \bfJ_\bfs \left( {\frac 1 {(\bfI \otimes (\bfe_{n_c} \bfe_{n_c}^{\top})) \exp(\bfs)}} \right)\\
&=& \nabla^2_\bfs E_1 + \nabla^2_\bfs E_2
\end{eqnarray*}
\end{frame}


\begin{frame}[fragile]\frametitle{Hessian of Softmax Function - 2}

First term is easy
\begin{eqnarray*}
\nabla^2_{\bfs} E_1(\bfs) &=& \frac{1}{n}{\rm diag} \left( {\frac 1 {(\bfI\otimes (\bfe_{n_c} \bfe_{n_c}^{\top})) \exp(\bfs)}} \right)
{\rm diag} \left( \exp(\bfs)\right) \\
&=&
\frac{1}{n}{\rm diag} \left( {\frac {\exp(\bfs)} {(\bfI \otimes (\bfe_{n_c} \bfe_{n_c}^{\top})) \exp(\bfs)}} \right) 
\end{eqnarray*}
\pause
\bigskip


Reshaped back, a matrix-vector-product with $\bfV \in \R^{n_c \times n_f}$ is

$$ \nabla^2_{\bfS} E_1(\bfS) \bfV = \frac{1}{n}  \left( \left( {\frac {\exp(\bfS)} {\bfe_{n_c}\bfe_{n_c}^\top \exp(\bfS)}} \right) \odot ( \bfV \bfY) \right) \bfY^{\top} $$

\end{frame}

\begin{frame}[fragile]\frametitle{Hessian of Softmax Function - 3}
	
$$
E_2 =\frac{1}{n} {\rm diag} (\exp (\bfs)) \underbrace{\bfJ_\bfs \left( {\frac 1 {(\bfI \otimes (\bfe_{n_c} \bfe_{n_c}^{\top})) \exp(\bfs)}} \right)}_{=:\bfT}.
$$
 Using chain rule, we get
 $$
 \bfT = - {\rm diag}\left({\frac 1 {\left((\bfI \otimes (\bfe_{n_c} \bfe_{n_c}^{\top} )) \exp(\bfs)}\right)^2}\right) (\bfI \otimes (\bfe_{n_c} \bfe_{n_c}^\top)) {\rm diag}(\exp(\bfs))
 $$

 \pause\bigskip

 After reshape the matrix-vector-product with $\bfV \in \R^{n_f \times n_c}$ is

 \begin{align*}
 \nabla^2_{\bfS} E_2(\bfS) \bfV=  - \frac{1}{n}\left( \frac {(\exp(\bfS))}{\bfe_{n_c}(\bfe_{n_c}^\top \exp(\bfS))^2 }\right) 
   \odot (\bfe_{n_c}\bfe_{n_c}^\top (\exp(\bfS) \odot (\bfV \bfY)))\bfY^{\top}  
 \end{align*}

\end{frame}


\begin{frame}[fragile]\frametitle{Newton-CG for Softmax function }

Mat-vecs with Hessian can be computed as
\begin{eqnarray*}
	\nabla^2_\bfW E(\bfW) \bfV&= \frac{1}{n}  \left( \left( {\frac {\exp(\bfS)} {\bfe_{n_c}\bfe_{n_c}^\top \exp(\bfS)}} \right) \odot ( \bfV \bfY) \right)\bfY^\top \\
	-&\frac{1}{n}  \left( {\frac {(\exp(\bfS))}{\bfe_{n_c}(\bfe_{n_c}^\top\exp(\bfS))^2 }}\right)  \odot (\bfe_{n_c}\bfe_{n_c}^\top (\exp(\bfS) \odot (\bfV \bfY)) )\bfY^\top
\end{eqnarray*}
(possible to further simplify this to reduce operations)

\bigskip
\pause

Now, ready to use matrix-free Newton method with Armijo linesearch and CG solver that computes
$$
	\nabla^2_\bfW E(\bfW) \bfD \approx -\nabla_\bfW E(\bfW).
$$

Remarks:
\begin{itemize}
	\item how well to solve? use large tolerance on relative residual
	\item can accelerate CG with preconditioning $\leadsto$ PCG
	\item possible to omit second term in Hessian?
\end{itemize}
\end{frame}

\begin{frame}[fragile]\frametitle{Coding: Hessian of Softmax Function}

Extend your softmax function, so that it returns a function handle computing mat-vecs with Hessian if needed.

\begin{verbatim}
function[E,dE,d2Emv] = softmaxFun(W,Y,C)

% Your code from before
if nargout > 1
% Your code from before
end

if nargout > 2

% Your new code here
d2Emv = @(V) ...;
end
end
\end{verbatim}
\begin{center}
	Don't forget to check your derivatives!
\end{center}
\end{frame}


\begin{frame}\frametitle{Project: Linear Classification for MNIST}
	 Write a code for solving

	\begin{align*}
	 \bfW^* = & \argmin_{\bfW} \quad - \frac{1}{n}\bfe_{n_c}^{\top} \left( \bfC_{\rm obs} \odot \bfS\right) \bfe_{n} 
	+ \frac{1}{n} \log(\bfe_{n_c}^\top \exp(\bfS))\bfe_n \\
	          & \text{subject to} \;\bfS = \bfW \bfY - \bfe_{n_c} \bfs
	\end{align*}
	
			and apply it to the MNIST test data set.
			
	\begin{enumerate}
		\item write a stable code for evaluating the softmax loss
		
		\item implement and test the derivatives we computed in class or use automatic differentiation to compute gradients and Hessians.
		
		\item use optimal weights to predict labels for test data. How well does your solution generalize?
		
		\item visualize the rows of $\bfW$ as images.
		
		\item compare with the results you got in the last module.
	\end{enumerate}

\end{frame}

\begin{frame}
	\frametitle{$\Sigma$ : Linear Classifiers}
	
	\begin{itemize}
		\item classification: use (multinomial) logistic regression
		\item compare to least-squares
		\begin{itemize}
			\item advantage: output is in unit simplex
			\item disadvantage: problem does not decouple into $n_c$ subproblems, statistical interpretation?
		\end{itemize}
		\item loss function: cross-entropy preferred (convex)
		\item can choose between many numerical optimizers: SD, Newton, $\ell$BFGS, ADMM,\ldots
		\item useful in supervised and semi-supervised tasks
		\item work well iff classes can be separated by hyperplanes
	\end{itemize}
\end{frame}

\begin{frame}[allowframebreaks]
	\frametitle{References}
\bibliographystyle{abbrv}
\bibliography{NumDNN}

\end{frame}

\end{document}
