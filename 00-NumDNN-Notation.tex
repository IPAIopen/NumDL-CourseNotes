\documentclass[12pt,fleqn]{beamer}


\xdefinecolor{lavendar}{rgb}{0.8,0.6,1}
\xdefinecolor{olive}{cmyk}{0.64,0,0.95,0.4}
%\xdefinecolor{olive}{cmyk}{1,0,0,0}
\xdefinecolor{mag}{cmyk}{0.1,1,0,0.2}
\xdefinecolor{lblue}{rgb}{0,0,1.5}
\xdefinecolor{lred}{rgb}{1,0,0}
\xdefinecolor{mine}{cmyk}{1,0,0.2,0}
\xdefinecolor{bluel}{cmyk}{0.1,0,0.9,0.4}

\usepackage{amsmath,amssymb,dsfont,mathrsfs}
\usepackage{tikz,pgflibraryplotmarks}
\usepackage{multimedia}
\usepackage{wasysym}
\usepackage{rotating}
\usepackage{algorithm,algorithmic}
\usepackage{graphicx} % more modern
\usepackage{subfigure}
\usepackage{booktabs}

\usepackage{pgfplots}
\usepackage{verbatim}

\usepackage{setspace}
\newlength\iwidth
\newlength\iheight

\newcommand\makebeamertitle{\frame{\maketitle}}%
\graphicspath{{./images/}}
\setbeamertemplate{navigation symbols}{}
\addtobeamertemplate{navigation symbols}{}{%
    \usebeamerfont{footline}%
    \usebeamercolor[fg]{footline}%
	\insertshorttitle
    \;--
    \insertframenumber
}

\newcommand{\sectionstart}{
	\only<beamer>{
 	\begin{frame}% (fold)
 		\begin{centering}\Huge \insertsection \par\end{centering}
 	\end{frame}% frame the_application (end)
	}
 }


% make bibliography entries smaller
\usepackage{natbib}
\setbeamertemplate{bibliography item}{[\theenumiv]}
\renewcommand\bibfont{\scriptsize}
\setbeamertemplate{frametitle continuation}[from second]
\newcommand{\tcr}{\textcolor{red}}
\newcommand{\tcrd}{\textcolor{red}}
\newcommand{\tcb}{\textcolor{bluel}}
\newcommand{\tcm}{\textcolor{mag}}
\newcommand{\tcg}{\textcolor{olive}}

\newcommand{\R}{\mathbb{R}}
\newcommand{\C}{\mathbb{C}}

% bold lower-case for vectors
\newcommand{\bfa}{{\bf a}}
\newcommand{\bfb}{{\bf b}}
\newcommand{\bfc}{{\bf c}}
\newcommand{\bfs}{{\bf s}}
\newcommand{\bfm}{{\bf m}}
\newcommand{\bfd}{{\bf d}}
\newcommand{\bfe}{{\bf e}}
\newcommand{\bfu}{{\bf u}}
\newcommand{\bfy}{{\bf y}}
\newcommand{\bfx}{{\bf x}}
\newcommand{\bfh}{{\bf h}}
\newcommand{\bfw}{{\bf w}}
\newcommand{\bfv}{{\bf v}}
\newcommand{\bfr}{{\bf r}}
\newcommand{\bfz}{{\bf z}}
\newcommand{\bfp}{{\bf p}}


% bold upper-case for linear operators
\newcommand{\bfA}{{\bf A}}
\newcommand{\bfB}{{\bf B}}
\newcommand{\bfZ}{{\bf Z}}
\newcommand{\bfM}{{\bf M}}
\newcommand{\bfC}{{\bf C}}
\newcommand{\bfD}{{\bf D}}
\newcommand{\bfQ}{{\bf Q}}
\newcommand{\bfJ}{{\bf J}}
\newcommand{\bfG}{{\bf G}}
\newcommand{\bfI}{{\bf I}}
\newcommand{\bfP}{{\bf P}}
\newcommand{\bfK}{{\bf K}}
\newcommand{\bfY}{{\bf Y}}
\newcommand{\bfW}{{\bf W}}
\newcommand{\bfR}{{\bf R}}
\newcommand{\bfL}{{\bf L}}
\newcommand{\bfF}{{\bf F}}
\newcommand{\bfT}{{\bf T}}
\newcommand{\bfS}{{\bf S}}
\newcommand{\bfX}{{\bf X}}
\newcommand{\bfU}{{\bf U}}
\newcommand{\bfV}{{\bf V}}
\newcommand{\bfH}{{\bf H}}

\newcommand{\diag}{\rm diag}

\newcommand{\calF}{\mathcal{F}}



\newcommand{\hf}{{\frac 12}}
\newcommand{\bftheta}{{\boldsymbol \theta}}
\newcommand{\bfmu}{{\boldsymbol \mu}}
\newcommand{\bfxi}{{\boldsymbol \xi}}

\newcommand{\bfLambda}{{\boldsymbol \Lambda}}
\newcommand{\bflambda}{{\boldsymbol \lambda}}
\newcommand{\bfSigma}{{\boldsymbol \Sigma}}
\newcommand{\bfepsilon}{{\boldsymbol \epsilon}}

\newcommand{\E}{\vec E}
\newcommand{\B}{\vec B}

\newcommand{\vu}{  {\vec {\bf u}}}

\newcommand{\grad}{  {\vec {\bf \nabla}}}

\newcommand{\lfrownie}{\textcolor{red}{\large{\frownie}}}
\newcommand{\lsmiley}{\textcolor{green}{\large{\smiley}}}

\newcommand{\curl}{\ensuremath{\nabla\times\,}}
\renewcommand{\div}{\nabla\cdot\,}
\newcommand{\divh}{\nabla_h\cdot\,}
\renewcommand{\grad}{\ensuremath{\nabla}}

\DeclareMathOperator*{\argmin}{arg\,min}


\title[Notations]{Notation}
\subtitle{Numerical Methods for Deep Learning}
\date{}

\begin{document}
\makebeamertitle

\begin{frame}
	\frametitle{Data}
	
	\begin{itemize}
		\item $n$ - number of examples
		\item $n_f$ - dimension of feature vector
		\item $n_c$ - dimension of prediction (e.g., number of classes)
		\item $\bfy_1,\bfy_2,\ldots,\bfy_n \in\R^{n_f}$ - input features
		\item $\bfY = \left[\bfy_1,\bfy_2,\ldots,\bfy_n \right] \in \R^{n_f \times n}$ - feature matrix
		\item $\bfc_1,\bfc_,\ldots,\bfc_n \in \R^{n_c}$ - output observations 
		\item $\bfC = \left[\bfc_1, \bfc_2,\ldots,\bfc_n\right] \in \R^{n_c \times n}$ - observation matrix
	\end{itemize}
\end{frame} 

\begin{frame}
	\frametitle{Neural Networks}
	
	\begin{itemize}
		\item $f(\bfy,\theta) = \bfc$ - model represented by neural net
		\item $\theta \in \R^{n_p}$ - parameters of model
		\item $\theta^{(1)}, \theta^{(2)}, \ldots$ - parts of weights. Division clear from context. Examples
		\begin{enumerate}
			\item $\theta^{(j)}$ are weights of $j$th layer.
			\item $\theta^{(1)}$ are weights for convolution kernel, $\theta^{(2)}$ are weights for bias
		\end{enumerate}
		
		\item $N$ - number of layers
		\item $\bfK$ - linear operator applied to features
		\item $b$    - bias
		\item $\sigma : \R\to\R$ - activation function
	\end{itemize}
\end{frame} 

\begin{frame}
	\frametitle{Optimization and Loss}
	
	\begin{itemize}
		\item $E(\bfY,\bfC,\bfW)$ - loss function parameterized by weights $\bfW$
		\item $\phi : \R^{k} \to \R$ - generic objective function
		\item $\theta^*$ - minimizer of a function, i.e.,
		$$
			\theta^* = \argmin_{\theta} \phi(\theta)
		$$
		\item $\theta_1,\theta_2, \ldots$ - iterates
		\item $\bfd, \bfD$ - search directions
		\item $\alpha$ - step size
		\item $\lambda$ - regularization parameter
		\item $\nabla_\bfx F$ - gradient, if $ F : \R^k \to \R^l $, then $\nabla F(\bfx) \in \R^{k\times l}$
		\item $\bfJ_\bfx F$ - Jacobian of $F$ with respect to $\bfx$, $\bfJ_{\bfx} F = (\nabla_\bfx F)^\top$
	\end{itemize}
\end{frame}

 \begin{frame}
	\frametitle{Linear Algebra - 1}
	\begin{itemize}
		\item $\bfe_k \in\R^k$ - vector of all ones
		\item $\bfI_k$ - $k\times k$ identity matrix
		\item $\kappa(\bfA)$ - condition number of $\bfA$
		\item $\sigma_1(\bfA)\geq\ldots\geq\sigma_k(\bfA)\geq 0$ - singular values of $\bfA$
		\item $\lambda_1(\bfA),\ldots$ - eigenvalues of $\bfA$
		\item ${\rm tr}(\bfA)$ - trace of square matrix, i.e., sum of diagonal elements
	\end{itemize}
\end{frame}

\begin{frame}
	\frametitle{Linear Algebra - 2}
	\begin{itemize}
		\item $\odot$ - Hadamard product
		$$
			\bfC_{ij} = \bfA_{ij} \cdot \bfB_{ij}, \quad \text{for} \quad \bfB,\bfA \in \R^{k\times l}
		$$
		MATLAB: $\texttt{C = A.*B}$
		\item $\otimes$ - Kronecker product
		$$
			\bfA \otimes \bfB = \left(
				\begin{array}{rrrrr}
					\bfA_{11} \bfB & \bfA_{12} \bfB & \dots & \bfA_{1l} \bfB \\
					\vdots         & \vdots         & \vdots& \vdots\\
					\bfA_{k1} \bfB & \bfA_{k2} \bfB & \dots & \bfA_{kl} \bfB \\
				\end{array}
			\right)
		$$
		MATLAB: \texttt{C = kron(A,B)}
		\item ${\rm vec}(\bfA)$ - reshape matrix $\bfA$ into vector (column-wise). 
		$$
		\text{Example:}\quad
			{\rm vec}\left(
				\left(
					\begin{array}{rr}
						\bfA_{11} & \bfA_{12} \\
						\bfA_{21} & \bfA_{22} 
					\end{array}
				\right)
			\right) = \left( 
				\begin{array}{r}
					\bfA_{11}\\
					\bfA_{21} \\
					\bfA_{12}\\
					\bfA_{22}
				\end{array}
			\right)
		$$
		MATLAB: $\texttt{a = A(:)}$
	\end{itemize}
\end{frame}


\begin{frame}
	\frametitle{Linear Algebra - 3}
	\begin{itemize}
		\item ${\rm mat}(\bfv,k,l)$ - reshape vector $\bfv\in\R^{kl}$ into matrix. $k,l$ omitted when dimension clear from context. Note
		$$
			{\rm mat}({\rm vec} (\bfA)) = \bfA.
		$$
		MATLAB: \texttt{V = reshape(v,k,l)}.
	
		\item ${\rm diag}(\bfv)$ - diagonal matrix with elements of $\bfv \in \R^k$ on diagonal
		
		MATLAB: \texttt{V = diag(v(:))}
		
		\item ${\rm diag}(\bfA)$ - diagonal matrix obtained by vectorizing $\bfA$
	\end{itemize}
\end{frame}

\begin{frame}
	\frametitle{Acronyms}
	\begin{itemize}
		\item CG - Conjugate Gradient Method
		\item VarPro - Variable Projection
		\item SD - Steepest Descent
		\item SGD - Stochastic Gradient Descent
		\item SA  - Stochastic Approximation
		\item SAA - Stochastic Average Approximation
		\item SPD - symmetric positive definite
		\item SPSD - symmetric positive semi-definite
		\item CV - Cross Validation
	\end{itemize}
\end{frame}
\end{document}






